\PoemTitle{Venezia / Torcello}
\label{ch:venezia_torcello}
\settowidth{\versewidth}{In chains, scrolled salamanders in runic fire.   Did Attila}
\begin{verse}[\versewidth]
Here is his throne, set in the trapezoid of barbed grass,\\*
Bordered by footpaths, two churches, a canal.\\*
It's sunset: the light falls in long spears,\\*
The brick church flares pink, dark red, blue shadows settle\\*
Inside.    The Virgin is tall \& narrow,\\*
Her eyes all pupil, like cat eyes.\qquad The damned\\*
Gaze from mosaic flames, gold \& iron blue\\*
And white.   Across millennia, twelve apostles stare\\*
With Byzantine severity: their robes fall\\*
In knife-blade folds. The Virgin's brazen Son\\*
Holds a scroll: All is gold, all.

Runcinate dandelion leaves border\\*
The throne's base.   Did he hold court here?\\*
Executions?   The sun is almost down.   The dusk\\*
Is green, the churches black.   Attila's throne is white,\\*
Luminous, \& shaped like a Victorian armchair.\\*
Once it may have been carved: saucer-eyed Roman youths\\*
In chains, scrolled salamanders in runic fire.   Did Attila\\*
Resemble the iron-blue God of the mosaics? Lion-\\*
Maned, His throne upholstered in dragon scales,\\*
Naked---except for a green loincloth?   The damned\\*
Are speared by angels.   The royal dusk

Deepens into emerald night.  Winding into the low brush,\\*
The undulant canal is a pale blue snake,\\*
Casting what little light there is\\*
Back toward the stars.\qquad We walk\\*
Back to the scalloped Adriatic, the slap-slap-slap\\*
Against stone piers.    It could be\\*
Any century. Behind us

There are no lights in the ragged field, dandelion\\*
Leaves border the black throne.
\end{verse}
