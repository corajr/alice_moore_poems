\PoemTitle{Murano / Glass Factory: Father \& Son}
\label{ch:murano}
\settowidth{\versewidth}{Roughly equine—averts his face, goes sidewise out}
\begin{verse}[\versewidth]
There are two workmen:\\*
One---young, not yet twenty, the lines of his body\\*
Roughly equine---averts his face, goes sidewise out\\*
A cement arch into another workroom.

The burly man remaining turns from his task.\\*
His eyes are faience, the eyes\\*
Of a fabulous marine creature: startled\\*
You recognize him: he is some relative

Or other of someone or other who is a friend\\*
Or nephew or cousin of yours. A few quick words,\\*
A brief genealogy, \& he hails you, glancing\\*
At me. In the midst of your reply

I hear \textit{americana}. I look away, toward the empty\\*
Arch. \textit{He will demonstrate for you}, you whisper,\\*
Nudging me. I watch as he dips into molten glass\\*
A long iron rod---a fuchsia globe comes away

Is spun out and almost in the same movement\\*
Plunged into another cauldron, enveloped\\*
In a colorless molten sheath.\\*
Quickly spinning the rod with one hand,

He grabs a pair of pincers with the other\\*
---They look medieval \& cruel---\\*
And tugs out a snout, an arched neck, legs---\\*
Lifts the legs into a prance, plucks

Ears upright, with a series of deft pinches\\*
Raises a mane \& tail \& cuts\\*
The whole off, hands it to me\\*
Wrapped in brown paper: the flame-pink heart

Of the hot glass colt\\*
Incandescent\\*
But cooling to the ultramarine\\*
Blue of the fabled

Adriatic.
\end{verse}
