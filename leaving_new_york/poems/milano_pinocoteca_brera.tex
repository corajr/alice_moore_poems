\PoemTitle[Milano / Pinocoteca Brera]{Milano\\ Pinocoteca Brera}
\label{ch:milano_pinocoteca_brera}
\settowidth{\versewidth}{The window, of course, is narrow, \& the view}
\begin{verse}[\versewidth]
GESU CRISTO ALLA COLONNA\\*
Bramante

Over his right shoulder\\*
A small window enclosing\\*
A delicate landscape, tinted blue\\*
And olive-green.\qquad There is a lake\\*
Or inlet, a pier, boats, silk tents;\\*
The far shore rises in a gentle slope\\*
Bordered by clumped trees on two sides;\\*
Beyond, mountains pale \& sheer\\*
As glaciers dissolve\\*
Into the ice-blue dawn.

What landscape is it?  You tell me\\*
The scenery of southern Italy\\*
Really looks like this: steep \& surreal\\*
Peaks, little patchwork fields\\*
Bordered by olive and cypress,\\*
Geometric ruins embroidered\\*
By wild herbs \& flowers,\\*
Jewel-like insects with glassine wings.\\*
Then tell me: who lives\\*
In those silken tents, perfectly mirrored\\*
In the lake's still waters?  Who steers\\*
The narrow boats?  Who gathers\\*
The olives?  Watches from the piers?

Is the earth really so still, so empty?\\*
The window, of course, is narrow, \& the view\\*
Distant. If there are women harvesting\\*
On that far slope, or men navigating\\*
The ethereal lake, they are obscured\\*
By the miniature scale,\\*
And by the distance between them\\*
And us.   The light falling across the lake\\*
---Or inlet---cannot reveal them.\\*
But it illumines the body of Christ, muscular\\*
And youthful, though his face is aged.

This Christ is clearly a man---\\*
His arms, shoulders, chest\\*
Those of a carpenter---\\*
His face is divided in half\\*
By ragged chiaroscuro.\\*
A rope is knotted around his neck\\*
Another cuts into the flesh\\*
Of his left arm: these bind him\\*
Bodily to the engraved column\\*
But his troubled gaze follows\\*
Something to our right.\\*
Is there a cry on his parted lips?\\*
What cry?
\end{verse}
