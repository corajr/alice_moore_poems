\PoemTitle{On First Looking into the New Oxford Rhyming Dictionary}
\label{ch:on_first_looking}
\settowidth{\versewidth}{Of mystery.          Meanwhile, what of that mate for ``orange''?}
\begin{verse}[\versewidth]
There is, of course, no rhyme for ``orange,''\\*
Except, as Garrison merrily instructs us, ``door-hinge.''

Where's Edward Lear when you need him?\\*
Probably versing with the heavenly seraphim.

If I turn to dictionary Oxfordian\\*
I am directed to choices smorgasbordian

For ``king'' and ``queen,''\\*
For ``be,'' ``bee,'' and ``seem''---

Dozens for ``Utrecht''\\*
For ``hunch-backed'' and ``hand-picked.''

Yet they've missed ``Holly-Go-Lightly,''\\*
A likely rhyme for ``glow brightly,''

Less so for ``impolitely,'' ``to spite me,''\\*
``Not so tightly!''    In Section 4.1 we see

The chimpanzee \textit{en famille}\\*
Eating ratatouille; in other entries

Words even more inauspiciously yoked\\*
Than brothers Koch.

(I'd love to see the poem in which ``Wittgenstein''\\*
Meets his match in ``infantine.'')

Oxford omits it, but please note the Scots ``hirple''\\*
A (not-so-lame) rhyme for ``purple'':

A pair that might suggest a victim Hebridean\\*
Limping into sunset antipodean

Hirpled and empurpled\\*
In land unpeopled

Galumphing dactyllically into the gloaming\\*
Unless you choose to send him motor-homing

Now trochaically\\*
And a bit more prosaically

Out of the poem entirely, and into sunlit prose\\*
Where explanations follow one, two, three, without undertows

Of mystery.\qquad Meanwhile, what of that mate for ``orange''?\\*
The apparently unpromising ``door-hinge''

Does yield a couplet both euphonic\\*
And, to birders like me, mnemonic:

\qquad \textit{The catbird, whose rump is dusty orange,\\*
\qquad Has a cry just like a rusty door-hinge.}
\end{verse}
