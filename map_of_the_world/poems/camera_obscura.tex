\PoemTitle{Camera Obscura}
\label{ch:camera_obscura}
\settowidth{\versewidth}{Where each day feels as if hurtling towards nothing.}
\begin{verse}[\versewidth]
I did not come easily\\*
To Vermeer.  Chagall was my love\\*
At sixteen. For Chagall

All possibilities seemed actual.\\*
Lovers flew over shetls, cafes, the Seine,\\*
In skies, Russian or Parisian,

Ultramarine from the tube, unmixed.  Blue-feathered\\*
Roosters, red mermaids, chrome\\*
Yellow goats with the wise eyes

Of prophets: all inhabited\\*
Worlds without the complication of gravity\\*
So essential to Vermeer's interiors.

But now I live in those rooms\\*
And see how vast they are,\\*
How the world, many worlds

Come to them, and in them\\*
In that impossible condensation,\\*
Everything is visible, yet nothing

Is flayed open, is named. \\*
That silence is a kind \\*
Of shelter, and I seek it.

Thus in middle life I have come\\*
To take Vermeer as my astronomer,\\*
My navigator.   He turns

His lens on my world,\\*
And at his side I scrutinize\\*
Not Hell, not Heaven
But the stillness between the tick-\\*
Tocks of an ordinary afternoon, an every-\\*
Day day on the same earth

Where each day feels as if hurtling towards nothing.\\*
There is relief only\\*
In the almost perfect geometry

Of these interiors, windows\\*
To one side, diamond-paned,\\*
\textit{The light oblique.}
\end{verse}
