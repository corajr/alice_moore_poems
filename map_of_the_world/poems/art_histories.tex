\PoemTitle{Art Histories}
\label{ch:art_histories}
\settowidth{\versewidth}{                                   Fire green, gold, carmine:}
\begin{verse}[\versewidth]
1.\\*
Everything is a matter of life \& death,\\*
Especially these old tales, ancient stories\\*
Of thorns \& blood, wishes \& journeys,\\*
Darkness behind the wrong door.

Darkness behind every door.\\*
When I walk at night\\*
There's no one on the stoops \& porches---\\*
Everyone's inside, the one window glows blue.

What of those kings \& queens carved in stone\\*
At the Royal Portal, Cathedral of Chartres?\\*
Kings, queens, virgins, prophets stand\\*
As if stilled by a vision of eternity: Desire,

Their faces say, is the gravity of this world,\\*
The light of the next.

2.\\*
                                   Fire green, gold, carmine:\\*
Night rain slicks the streets, slicks the darkness.\\*
Dark leaves shine as they shift under street light,

Darken as the breeze dies. Downtown\\*
Traffic signals on the parkway trail blistering comets\\*
Like the blood of saints, radiant spoor \\*
Spilled where every driver gazes \& none sees.

3.\\*
On the cusp of the Dark Ages, did the stonecutter\\*
Gaze into the face of his stone queen \& know\\*
What she looked upon, even if he could not say?\\*
Journeys \& wishes, wishes granted,

Desire proving fatal at every turn.\\*
Fatal \& vital, a Boschian paradise\\*
Dragging its white roots through Hell,\\*
Blossoming through the body, as the body:

Its necessities \& their transcendence\\*
Both.

4.\\*
         Watching the night rather than anything\\*
In it, yearning after its extremes, its purity.\\*
Light \& dark---and we say, Life and death.

Stillness shot through with restless motion.\\*
Someone always wanders the dark streets\\*
Someone gazes at the pools of street light\\*
And at the objects that hover on the margins

Smouldering in dim profile. How\\*
To anatomize the heart that flutters, that stills\\*
At what rises in its core? At the dark's\\*
Lovely offerings, velvety, plumed, obscure?

5.\\*
No, night is not death. Black\\*
Is not death, black is harmony, Nevelson instructed \\*
Her Manichaean interviewer. Death\\*
Is the light contracting in the eyes of the killer

The moment before he learns what it is\\*
He will do. Or afterwards, the woman who stares\\*
Into the lens without shame, in the cold\\*
Fever of knowing that to cross over

Is as vain as to return, now she is\\*
So steeped in blood:\\*
``They didn't cry because they knew us.\\*
They just made big eyes.

We killed

Too many to count.'' Death comes to us all,\\*
Sometimes in our own hands, a hellish bolt\\*
Coruscating like God's sword through the heart\\*
We thought we knew: light we dread.

6.\\*
So light sears \& scorches \& night comes\\*
As a relief: Cool air on the skin\\*
Like a lover's tongue, the constellations\\*
Unstrung, new stories, new deities

Rising in a slow spiral from the old\\*
Chaos: chaos resolving into order only\\*
Apparently, only for a moment, only\\*
Amid the thorns \& blood of half-remembered tales,

Or under the glance of this tessellate\\*
Virgin, her iron-blue eyes made fierce\\*
By our desires: Byzantine, framed\\*
By our hands, her lineaments ours

For the moment, in the moment we return\\*
Her gaze.

7.\\*
                 Incarnadine: our hands\\*
Blood-red, reddening all they touch. Fatal\\*
And vital: in every stroke of Van Gogh's 

Last works: love, desire, despair, the heart\\*
Riven by what it embraces. For Van Gogh\\*
A year of fire, cold \& blue, coming to him at midnight,\\*
Coming to him at dawn. Crows black glyphs,

The language of the last days, incised\\*
On the flame-blue sky of noon: There he burns,\\*
Thus he speaks, as if alive. The silence\\*
Is terrible \& beautiful.

8.\\*
                                        Blake's tyger

Roams Van Gogh's night sky. In that furnace\\*
Is his heart. In his heart is ours, molten,\\*
Convulsed, in its pinched \& human dark.\\*
Blake's tyger is Blake's creator.

His jaws yawn to swallow us\\*
Each alone, in unanswerable\\*
Solitude, destruction the fearful\\*
Harmony counterbalancing all creation

Always.      Is this\\*
The root of desire, even love?  The tyger's\\*
Terrible fire consuming the dark\\*
At the core?

Coda:\\*
Giotto's ``Lamentation,'' Padua

Giotto's Mary kneels. She knows\\*
What she looks on: her Son's face pales\\*
And his features grow leaden.\\*
The complete solitude of the final divide

Settles upon them. She cannot whisper to him\\*
Across this gulf, and her love\\*
Strands her in grief.\\*
Her eyes darken,

And the anguish in them is unanswerable.\\*
Even the angels tear their hair,\\*
Writhe at the sight\\*
Of blood \& thorns, the gravity of death.

The body is drawn to earth, but always\\*
Overhead the sky arches, a goddess,\\*
Star-pierced, a cyclone, a horror, an infinite\\*
Gulf into which we can never quite fall.
\end{verse}
