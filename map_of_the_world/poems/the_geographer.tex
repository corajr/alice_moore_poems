\PoemTitle{The Geographer: Vermeer}
\label{ch:the_geographer}
\settowidth{\versewidth}{Perhaps, of the painter, yet it does not matter}
\begin{verse}[\versewidth]
Probably modeled on van Leeuwenhoek, the Delft\\*
Microscopist, navigator, astronomer, \& friend,\\*
Perhaps, of the painter, yet it does not matter

That Vermeer's geographer is a man,\\*
And I a woman. After all this time\\*
I enter his skin \& feel the same

Light on my face. Now what\\*
Questions can I ask? What can I not\\*
Imagine asking? The dividers measure

Distances on vellum maps, maps \\*
That tell how far men have gone, some\\*
Dying at the sight of unknown shores.

But others don't know what they see,\\*
Think it known, \& conceive it\\*
In terms of gold, silk, spices.

Measured this way, the new world\\*
Becomes familiar, \& if, oddly,\\*
Still exotic, it is the exoticism

Of legends \& histories already centuries\\*
Old.  Such a world is mapped\\*
Before it is seen. But the man in the light

Of this window, the man I am\\*
For the moment, knows nothing\\*
Of this, knows

How little he knows, \& wonders\\*
At how this cobbled street, \\*
Pearled with ordinary light,

Somehow lies connected to the pathless routes\\*
To, and across, new continents.\\*
Now the light in my face---his face---

Comes from within. Here,\\*
On an unnumbered, unremembered day\\*
In Delft, in Holland, in the seventeenth

Century of the current era.
\end{verse}
