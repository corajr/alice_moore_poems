\subsection*{Acknowledgements}

``y = sin x'' and ``Punctuation'' were published in \textit{The Dirty Napkin}, Volume 4, No. 1.

``Evening Class'' was published in an earlier version in \textit{The Amherst Review}, Volume IX, 1981.

``Main Street Goes Up in Flames'' won an honorable mention in the 2013 Troubadour Poetry Competition and was published on their website.

\subsection*{Art Histories (p. \pageref{ch:art_histories})}
c. 1994-2004. These 8 poems, plus coda are an extended meditation on too many themes
for me to sum up neatly. The Hutu-Tutsi conflict and massacres turn up
(``We killed too many to count'' is a quote taken from the radio); ``Black
is harmony, not death'' is paraphrased Louise Nevelson, the sculptor; the
art history class I was teaching the summer of 1994 is the spine of the
poem; ``incarnadine'' comes from Macbeth: Van Gogh is there because he and
Blake cross-illuminate each other, and because Van Gogh is one of my
favorite artists; Blake's furnace of creation AND destruction is central
to the poem: always some destruction, some chaos, in creation. And a
quote from Alexis opens it: she said to me once that for me, it seemed
everything was a matter of life and death.

Well, now everything IS a matter of life and death.

\subsection*{Lives of the Saints/Lives of the Terrorists (p. \pageref{ch:lives_of_the_saints})}
c. 1998-2000. An obvious play on the opening of Art
Histories. The imagery in section 5 tells me I was writing this around
the time we went to Scotland in 1998; my dates here are more guess
(based on that) than anything else

The religious imagery (section 6) of Giotto's frescoes in Padua as usual
does not signify religious feeling on my part, but rather my absorption
in the humanity of Giotto's portrayal of human beings---he's like
Shakespeare: his "types" are complex, layered, felt. And none of them
are bored like the 8 o'clocks.

\subsection*{Can You Imagine a Hotel Called the Capri in Montana? (p. \pageref{ch:can_you_imagine})}
c. 1994-1999. Part of a group of poems written about
traveling back to the east coast from Yakima. Any Hotel, Any Night is
another. But of course the poems are not just about traveling: this is
about perception, and what we can know about other people, even those
closest to us. Ironic that what I don't know about my "daughter" won't
come out for another ten years: assigned female at birth, she's male.
But that was a truth that at that time I couldn't see and Alex couldn't
articulate.

\subsection*{Leaving Butte, Montana (p. \pageref{ch:leaving_butte})}
c. 1993-95. Another poem about driving east, from
Yakima---it connects with Flight, etc.

\subsection*{Segue, Elision (p. \pageref{ch:segue_elision})}
c. 1993-1995. One of the group that includes Flight and
Leaving Butte Montana: driving across country from Yakima to Corning.
Harold and the Purple Crayon: sometimes the world and your own
imagination allow you to find just what you want or what you imagine
that you wanted: there seems a natural connection. But sometimes there's
an elision: something is missed, not seen, lost.

\subsection*{Any Motel, Any Night (p. \pageref{ch:any_motel})}
Probably composed 1993-1995. Based on my return from
Yakima and the cross-country trip I took with Alex, then Nena. The war
in the Balkans was nightly news fodder. It haunted and horrified me.

I've debated over the years whether to change "daughter" to "son" in
this and other poems, but as they are written from my perspective, I
can't see changing it. I don't know how I would've experienced it if
Nena had been John (for Bob's friend John who committed suicide). Alex
was born just five years after my mother died, and my feelings about her
were heavily intertwined with some quasi-mythic ideas I had about my
mother-daughter bond with my mother being reconfigured in my bond with
my daughter. There was no attempt to actually produce that
relationship---I was aware, even then, of its problems---but I felt
part of something larger. If Alex had been assigned male at birth, I
have no idea what deeper themes that would have been suggested to me.
Maybe startlingly similar. But I just can't excise what's there, and
what was true for me, for a guess at a different
reality.

The ``awful, tasteful painting'' is Elizabeth
Bishop's. A moment's tribute to her.

\subsection*{Flight (p. \pageref{ch:flight})}
c. 1993-1995. This ties in with Any Hotel, Any Night and
Can You Imagine a Hotel (etc.): it is part of the series of poems that
explore the experience of driving cross-country with Alex, from Yakima
to Corning. Thnis particular poem is haunted by Any Hotel, Any Night, by
what the mother and daughter witness on the TV.

\subsection*{Main Street Goes Up in Flames (p. \pageref{ch:main_street})}
c. 1994-1995. Another of the group that includes Flight
and Can You Imagine a Hotel etc. The observation that the men drinking
coffee have "eyes the color//Of sky rippling on ditchwater" is not meant
as an insult. Their eyes are blue, and there's a certain intimacy
between them and the land they know so well.

But Alex and I were certainly out of place there.

\subsection*{Desert Music (p. \pageref{ch:desert_music})}
c. 1992-1994. Early evidence of my preoccupation with
Shakespeare's Lear. Here I'm very loosely identifying with him, his
rage, his inability to control---or contain---his
emotions.

\subsection*{Lear's Daughter  (p. \pageref{ch:lears_daughter})}
c. 1992-1994? Another early Lear poem. This one using
Lear to explore the terror of living with my father when I was an
adolescent and teen, searching for refuge.

\subsection*{Cordelia's Daughter (p. \pageref{ch:cordelias_daughter})}
c. 1992-1994. Alex was (to all appearances) still my
daughter at this point, and given her articulate defiance of all sorts
of conventions, I found myself thinking about the usual conflicts in
store for articulate, defiant women. But I could at least foresee that
she would never become inarticulate, and would defy the obstacles thrown
at her.

\subsection*{Landscapes (p. \pageref{ch:landscapes})}
c. 1992-1994. There's a bit of Tom Waits in here, a song
of his from Bone Machine, called Black Wings. It was written in the the
despair I felt over an unrequited love, one that I now feel lucky to
have escaped, because I think this man didn't want requited love, though
he exuded the despair of a man who longs to be loved. I remained friends
with him for ten years after leaving, but once I didn't keep initiating
contact, he dropped off the edge of the earth. Far too one-sided.

\subsection*{Naturally Death (p. \pageref{ch:naturally_death})}
c. 2008-2010. One of my favorite poems of mine. And a
good use of the imagery provided by my two-hour daily
commute.

\subsection*{y = sin x (p. \pageref{ch:y_sin_x})}
c. 2008-2010? Not written to commemorate any specific
deaths, as one person assumed---she found it "moving." I don't find it
moving---I think it's rather clinical in tone. The crosses were real:
on a curve on my daily commute to CCC, they seemed to commemorate some
deaths that occurred there, but I knew nothing more about them, and by
the time I noticed them, they looked a bit weathered. The title is the
formula for a sine curve: perfect in form, beautiful, inhuman.

\subsection*{Anatomy (p. \pageref{ch:anatomy})}
Composed about
2001-2003, when my marriage to Steve was breaking up. I don't think it
requires further explanation.

\subsection*{Out of This (p. \pageref{ch:out_of_this})}
c. 2000-2003. How tragedy and loss descend out of
nowhere.

\subsection*{Ancient Music, Perhaps  (p. \pageref{ch:ancient_music})}
Composed probably late 90s,
early 2000s. Most likely early 2000s, because its subject is
sadness.

Sadness is everywhere in the world. But
there's also death, and the persistence of art that preserves.

\subsection*{Fossil Life (p. \pageref{ch:fossil_life})}
c. 2000. One of my favorite poems. It's about the
trilobite fossil I bought at A2Z, the science store in Northampton. Of
course it's about much else besides.

\subsection*{Jurassic Lagoon (p. \pageref{ch:jurassic_lagoon})}
c. late 1990s, early 2000s? I love my dinosaur poems.

An important theme in this one is the question of how we reconstruct
what we've never seen, never can see; how what we know shapes what we
don't. But close observation and logic~ tell us so much: traces of
chemicals tell us about pigment and the likely colors of the dinosaurs.

\subsection*{Mythological Bird (p. \pageref{ch:mythological_bird})}
\subsection*{Petrified Forest (p. \pageref{ch:petrified_forest})}
c. 1992-94. Thin White Rope's lyric about the Hunter's
Moon made its way in here. Also my love of dinosaurs, and my perpetual
sense of isolation.

\subsection*{Reading Before Sleep (p. \pageref{ch:reading_before_sleep})}
c. 2002-2008. Reading Will his favorite "disaster"
books---volcanoes, tornadoes, earthquakes; reading about strangle
creatures and unfamiliar places (planets, stars,
galaxies).

\subsection*{The Age of Dinosaurs (p. \pageref{ch:the_age_of_dinosaurs})}
Probably written in the 1990s. Part of the question here is, Is it necessary to have an awareness of
life and death in order for a death to be tragic? Must there be a sense
of purpose? And when life is constructed with less complicated
consciousness---or a differently organized consciousness--- what is
awareness like? How much of it remains in our so-called "reptile
brain"?


\subsection*{Triassic Sandstone (p. \pageref{ch:triassic_sandstone})}
c. Late 1990s, early 2000s. One of my favorite dinosaur
poems.

\subsection*{A Map of the World (p. \pageref{ch:a_map_of_the_world})}
c. 2000-2003?

Me picking up every scallop shell on
Martha's Vineyard.

Ideas of order and patterns tend to impose themselves on the world.

\subsection*{\textit{Thatcheria Mirabilis} (p. \pageref{ch:thatcheria_mirabilis})}
c. 2010-2012. One of my favorite poems, partly as a
poem, and partly for the shell, the miraculous thatcher shell. In the
Golden Guide I had, The Seashells of the World, this shell fascinated
me, and I never thought to actually own one. Now I do, and my
fascination with its sculptural beauty has not
lessened.

\subsection*{Physics Lesson (p. \pageref{ch:physics_lesson})}
c. 1998-2000. I love this poem. And it's got Will at the
heart of it, because he was this lovely little boy when I wrote
it.

\subsection*{The Second Law of Thermodynamics (p. \pageref{ch:the_second_law})}
c. 1998-2002. One of my favorite poems. Haunted by
Yeats' Lapis Lazuli. Does the 2nd law always win? Not til the heat death
of the universe. But even in this speck of dust, as long as artists haul
together the fragments of what's destroyed, there's creation, there's
something new.

\subsection*{Punctuation (p. \pageref{ch:punctuation})}
c. 2008-2010. The dope of the first two lines was Steve
Wilson, my jackass "colleague" at CCC.

The mediocrity and lack of imagination in some of my colleagues made me
wonder if they ever actually read or did anything creative.

\subsection*{Evening Class (p. \pageref{ch:evening_class})}
c. 1980-1981: some later rev. One of my oldest poems,
and published in the Amherst Review. Also one of my favorites.

\subsection*{A Smooth Stone (p. \pageref{ch:a_smooth_stone})}
c. 1978; 1985 and later, some revision. Poems this old
I'm often more critical of, but I love this poem just as it is. Walking
on Ogunquit beach in the evening.

\subsection*{Camera Obscura (p. \pageref{ch:camera_obscura})}
Probably late 1990s or early 2000s. This poem is in part
about the evolution of my taste in art. In high school I hated Van Eyck
and Vermeer---the technicolor of Chagall was much more appealing.
Stillness was not something I needed until I got
older.

\subsection*{\textit{The Geographer}: Vermeer (p. \pageref{ch:the_geographer})}
c. late 1990s or early 2000s. In part about how I do not
necessarily feel barred from entering the consciousness of a narrator,
painting of a person, etc., just because that person is presented as
male. It's also about expectations and how they shape
perception.

\subsection*{\textit{Woman in Blue Reading a Letter} (p. \pageref{ch:woman_in_blue})}
c. Late 1990s-early 2000s? Another poem on how Vermeer
speaks to me.

\subsection*{The Cold: Four Paintings and a Failed Taxonomy (p. \pageref{ch:the_cold})}
c. late 1990s/early 2000s. These five poems are really
part of one entity, and don't make complete sense without each other.

\subsection*{Magic Foxes Assembling at Oji (p. \pageref{ch:magic_foxes})}
c. 1984-1985. I love this woodblock print by Hiroshige,
and I love my poem about it. Of course it's also another poem about how
beliefs shape perception.
