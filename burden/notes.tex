\subsection*{Burden (p. \pageref{ch:burden})}
c. 2011-2013. Inspired in part by all those songs in
which a bird---a swallow or a sparrow---can escape---or abandon---a
relationship, a lover or some dire situation: remain free! Don't be
bound to a painful situation! Be free! But here the one left behind
mourns someone who has slipped into another realm--- death, most
likely, for ``nothing mortal can follow.''

\subsection*{Palimpsest (p. \pageref{ch:palimpsest})}
c. 2010-2012. I don't believe in spirits or ghosts, but
we do carry the people we've loved with us, and even after they've died,
that ``spirit'' finds its niches in the world.

\subsection*{December 22nd (p. \pageref{ch:december_22})}

c. 2010 or after? A bit of Keats' urn makes it in
toward the end here; also Sappho's poem ``He is more than a hero\ldots''
The flower the snow makes of the landscape comes from one of my earliest
love poems, written for the boy/ young man I had the crush on in
Professor Alspach's Modern Poetry class. That crush has been a sort of
muse for me, inspiring a good deal of reflective thought, but as I've
written elsewhere, no regret. He was a fantasy, but he was also a
person, with an actual personality and world view that I think would
have diverged from my own. But he made an excellent muse.

\subsection*{Constellation (p. \pageref{ch:constellation})}
c. 1970s? Revised since.

Set in Northampton. I estimate
a late 70s date of composition; could be early 80s; some of the details,
like the solo saxophone of Marion Brown, make me think this was written
when I was working in the kitchen at Packards and walking home late at
night, 1 or 2 in the morning. I think that was late 70s.

\subsection*{Etching (p. \pageref{ch:etching})}
c. 1980. You can see the connections to December 22nd
and Etching---the giant white flower; Orion.

\subsection*{Neil Young Country (p. \pageref{ch:neil_young_country})}
This might date back to the late 70s, I really don't know. There's been
some revision since.

I used to stay up all night when I
visited my home in Sudbury. Chuckie had taken possession of my bedroom,
so there was only the couch for me, and I was
restless.

\subsection*{Sidereus Nuncius \textit{[formerly ``Internal Injuries'']} (p. \pageref{ch:siderius_nuncius})}
c. 2000-2014. A lot of revision.

Every time someone dies, especially in an accident, perhaps, you wonder if it could have
been avoided. You want to replay simple decisions that left the person
vulnerable, you want to tweak timing, judgments, emotional pitch---and
you want a reason where there is none. Struggling with the
incomprehensibility of death is essential to this poem.

\subsection*{Ten Lines for Tom (p. \pageref{ch:ten_lines})}
2003. Poetry isn't Tom's thing. Kind of a pointless
exercise to write him a love poem---but it was a natural outpouring at
the time, and I like it as a love poem.

\subsection*{Sonnet (p. \pageref{ch:sonnet})}
c. 2003-2004. The second love poem I
wrote for Tom.

\subsection*{Snow Globe (p. \pageref{ch:snow_globe})}
c. 2004. Will, sledding here in the backyard. Me wanting
to always protect him from all hurt or harm, and knowing I
cannot.

\subsection*{Subduction (p. \pageref{ch:subduction})}
c. Late 1990s-2002. The last gasp of that unrequited
love.

\subsection*{Out of the Night (p. \pageref{ch:out_of_the_night})}
c. 2000-2003: rev. off and on for years; this is where it
arrested. I felt pretty bleak.

\subsection*{Fever (p. \pageref{ch:fever})}
c. 1985. I like this hallucinatory poem. Partly inspired
by the last time I took codeine, in the university's infirmary, where
I'd gone because I had a migraine. Codeine makes me have waking
dreams/hallucinations, and eventually it makes me throw up. Another
contributor to the poem was the French movie "The Woman Next Door," by
Truffaut. The violent passion of the movie resonates with the violent
feelings in the poem.