\section*{Death}

\subsection*{At Dawn (p. \pageref{ch:atdawn})}
My mother dying in the Intensive Care Unit at UMass Medical in Worcester. She
had a tube in her mouth that would have prevented speech, but I don't think she
was capable of speech: she was pretty heavily drugged. I think she knew we were
there, and could feel the touch of our hands and hear the murmur of our voices.
But there was a great deal of uncertainty and little room for comfort. Probably
each of her children took away a different version of our mother from that room,
and my sisters at least believing they'd see her again in heaven. I've carried
her in my heart ever since, but I will not see her again.

\subsection*{Christmas Light (p. \pageref{ch:christmaslight})}
There's light at Christmastime, but the darkness in January and February seems
all the inkier. In the wake of my mother's death, the world seemed toothed, and
at my mother's funeral, even the birds seemed mechanical and devouring. The
``splinter of glacier'' owes its origin to Hans Christian Andersen's long story
``The Snow Queen.'' The splinter of the shattered mirror causes you to see the
evil in others.

\subsection*{A Meditation on My Mother's Death (p. \pageref{ch:ameditation})}
As I've remarked elsewhere, no belief in God implied here. The ``hand of God''
is a veil, an illusion: I'm asserting a connection with my mother in the wake of
her death---but a connection that will not always be able to feel after all, as
poems like ``Three Dreams'' and ``Inferno'' make clear. The mother-daughter bond
theme is evident here as well.

\subsection*{Inferno (p. \pageref{ch:inferno})}
Based on the opening of Dante's poem, of course. My struggle with the
internalized spirit of my mother often puzzled me: she seemed to have turned
away from me, and that's one reason I began to explore the theme of the
connections between mothers and daughters. When Alex was born, apparently
female, it gave me a renewed sense of connection to being a daughter of my
mother, though there was still frequently the sense that my own mother, in
dying, had withdrawn from me. When I wrote this poem, I still had no idea
whether I was having a boy or girl (no amniocentesis then)---and ironically,
when Alex was born, I still didn't know, though we all assumed we did.

\subsection*{Death in Life (p. \pageref{ch:deathinlife})}
In my mother's day especially, how many women, in becoming mothers, buried their
own lives? I used to have the feeling that my mother's life before us was a
mystery, part of some other world. Now I wish I could ask her questions about
it, but then I didn't even know what questions to ask. Rather a cruel parallel I
draw here, between a child growing inside her, and cancer, but I think of how
having six children in slightly more than nine years left her unemployed (no
life at work, no work friends!) and isolated in a suburban neighborhood with no
possible friends among the neighbors. And no companionship from my father,
either, who, when he wasn't in a rage, was working two jobs and preoccupied with
those. I became very close to my mother partly because she needed me as a
companion. I think that's why she used to let me stay up past 9 p.m. and watch
TV with her: she was dreadfully lonely.

\subsection*{Nightfall (p. \pageref{ch:nightfall})}
Nightfall is, in a sense, death here, and in following my mother to the brink of
that darkness, we peered at death.

\subsection*{Three Dreams (p. \pageref{ch:threedreams})}
1. My mother returning in a dream, but still not quite connecting with me----
still dead, after all. But I look to the possibility that if I bear a daughter,
named for her, in a sense (Rosaleen = Elena Rose), it will ease her seeming
distance and alienation.

2. I did dream this, and in the dream, my mother rummaged in a junk drawer in my
kitchen, and found an old door knob, with the rounded knob impaled on a steel
shank that would have been screwed into the corresponding knob. After cutting of
my foot, she inserted the shank into the open end of my tibia, declaring---or
implying through her evident satisfaction---that it was just as good. This
actually strikes me as funny (though no humor is implied in the poem), because
my mother was so good at ``making do,'' and it just seems typical of her. In the
poem, this takes a darker turn, because she is passing on damage that she ought
to have protected me from, damage from my father, who was not just physically
crippled (it was the term we used growing up), but emotionally crippled, too.

3. I do have some good memories of my father---him taking us to the beach, for
example, or reading us his version of children's bedtime stories: Gogol's ``The
Nose,'' or Kafka's ``Metamorphosis'' or ``The Penal Colony.'' But I mostly
remember being terrified of his rages and violence. His anger could explode out
of nowhere---there was no way to predict what might set him off, and so I lived
in a constant miasma of fear. Day to day, week to week, year after year, until I
left home at 18.

\subsection*{Requiescat (p. \pageref{ch:requiescat})}
Three or four years after my mother's death: a memory of her visit with me in
Northampton around Easter. Her lack of cynicism, and her reminder that here is a
place for child-like glee in the world. In the memory, she comes back to me
without any of the alienation of ``Three Dreams.''

\subsection*{Spring Returns (p. \pageref{ch:springreturns})}
The death of my mother's mother, 26 February 1948. I think in fact that my
mother took care of her mother in her apartment. But I'm not sure, and besides,
this is a poem, not a document. I don't know if there were nurses who ``speckled
the hospital lawn/like doves.'' But the image does what I want it to---that it's
somewhat unreal is all to the good: the approach to the death of someone you
love desperately is always a bit unreal. The Virgin at the end is of course the
Virgin Mary, but I was thinking of some of the fiercer images of her, as the
Byzantine image of her in one of the churches on Torcello. As fierce and
unforgiving as death itself.

\section*{Life}

\subsection*{Emergency at Birth (p. \pageref{ch:emergency})}
When Alex was born, and assigned female at birth, I had a daughter who seemed to
reestablish the mother-daughter bond that had been so troubled since my mother's
death five years earlier. Yet Alex nearly died at birth: Dr. Goldstein later
told me she had never seen a baby so ill at birth survive. Yet Alex grew into a
vigorous child with very much a mind of their own. I thought frequently of how
much my mother would have loved them. But it was a long time before I lost the
feeling that Alex had been snatched from death.

\subsection*{For Elena (p. \pageref{ch:forelena})}
A brief meditation on the intimacy of breastfeeding, and the inevitable formation of the child's self, and separation.

\subsection*{A Mother Watching her Daughter (p. \pageref{ch:amotherwatching})}
The quote at the beginning of stanza II is from Jane Eyre. And the mad king?
Most likely my father, always raging, always threatening. But there is a green
world, and its verdancy promises life, and a different path for my (apparent)
daughter than my own. And that infant, then child, is more defiant than I
managed to be, though a certain degree of defiance came out in my being such a
tomboy---I knew it bothered my father, and I was glad to be able to defy him in
this way at least. A fool ``Such as teased Lear'': fool's were not fools, and
were given a wide berth when it came to speaking truth to power, as we like to
say. And of course, any child ``teases [you] with questions'': they take you
into their ``brave new world'' and your own sense of wonder increases.

\subsection*{Child (p. \pageref{ch:child})}
Formerly ``Daughter.''

A further meditation on how the child's identity is
constantly separating from the parent's conception of it; further, that any
``names and forms'' we impose are just that: imposed. Especially true with Alex.

\subsection*{On Nena's Drawing (p. \pageref{ch:onnena})}
Now the child's drawing reveals something of how they conceive the parent.

\subsection*{Darkness Visible (p. \pageref{ch:darknessvisible})}
The headnote points to this poem's origin in William Styron's book on his
depression: Darkness Visible. I think I wrote this when Alex was in middle
school or early high school, and the pressures of being a closeted transman were
becoming unbearable. He hadn't come out to me yet, and I knew he was troubled
and depressed, but I didn't know anything of the origins of his crisis. I knew
that my own teenage crises had been dark enough that I felt that I couldn't
confide in anyone---but as a parent, you always feel you should know something,
say or do something to heal your child's heartache.

\section*{Hallucinatory Poems of My Mother In The Intensive Care Unit}

For months after my mother's death, the fact of her death was ungraspable to me:
I was used to not seeing her everyday, so her physical absence did not seem odd.
But as soon as I contemplated how unreachable she was---that I could not touch
her or talk to her---my mind swerved into incomprehension. These poems---``Often
Believed to Represent the Death''; ``Last Silences''; ``Near Death in the
Intensive Care Unit'' and ``No Time, No Place''--- have what is I think of as a
surreal or hallucinatory quality that, for me, catches that quality of
disbelief, of fractured reality that does not add up, cannot be reassembled into
a coherent whole. There's one more poem I'm including here, after slight
revision. It's in my mother's voice: her own vision of her death, and her vision
of all things tending toward where the Second Law sends them, in ``The Knowledge
of Death.'' The vision of the second half has a touch of the hallucinatory about
it.

\subsection*{Often Believed to Represent the Death (p. \pageref{ch:oftenbelieved})}
\subsection*{Last Silences  (p. \pageref{ch:lastsilences})}
\subsection*{Near Death in the Intensive Care Unit (p. \pageref{ch:neardeath})}
\subsection*{No Time, No Place (p. \pageref{ch:notime})}
\subsection*{The Knowledge of Death (p. \pageref{ch:theknowledge})}