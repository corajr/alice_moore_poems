\PoemTitle{Darkness Visible}
\label{ch:darknessvisible}
\begingroup
\setlength\epigraphwidth{9cm}
\epigraph{By this time it was early February, and although I was still shaky I knew I had emerged into light. I felt myself no longer a husk but a body with some of the body's sweet juices stirring again. I had my first dream in many months, confused but to this day imperishable, with a flute in it somewhere, and a wild goose, and a dancing girl.}{\textsc{William Styron}}
\endgroup

\settowidth{\versewidth}{The forest's edge, where saplings \& cornfields}
\begin{verse}[\versewidth]

Silly goose, the common and affectionate\\
Slander with which parents all nick-\\
Name their children, I called you.\\
But you were wild, not silly, and your heart

Writhed in its captivity, the clipped flights,\\
The known territory, the stories that all\\
Came to a conclusion. Every ending\\
Was a diminishment. The wild geese surged

And left. The cold and flame\\
Blue sky emptied.  Then winter came,\\
Years of winter, and winter's\\
Dark. Drained by the arctic

Isolation, you lost speech, lost even\\
The elemental cry of the wild\\
To its kind. Nor could you hear me.\\
The dancing girl of your childhood, the fluting

Reeds at summer's end, grew still,\\
Silence settled the world between us. What\\
Word would cleave that abyss?\\
Could I, should I, have split open

The darkness like a husk to bring you\\
The forest's edge, where saplings \& cornfields\\
Border each other, seemingly serene\\
In the ancient sunlight, no new

World, but the familiar one all\\
Glisten and enigma?
\end{verse}
