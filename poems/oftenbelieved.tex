\PoemTitle{Often Believed to Represent the Death}
\label{ch:oftenbelieved}
\settowidth{\versewidth}{Here an almost nude woman, wounded in the throat---}
\begin{verse}[\versewidth]
An even more beautiful example of his wildness is the long panel representing a Mythological Scene (fig. 1), often believed to represent the death of Procris, daughter of Erectheus, King of Athens. According to Ovid, Procris was pierced through the bosom by a javelin thrown by her husband, Cephalus, who mistook her for an animal concealed in the forest.  Here an almost nude woman, wounded in the throat, is mourned by a half-comprehending satyr, whose grief is as touchingly represented as is the wordless sympathy of the dog.\\
Frederick Hartt, History of Italian Renaissance Art, page 423, on a painting by Piero de Cosimo

1.

Here a half-comprehending satyr, whose grief---\\
Whose grief,\\
Uncomprehending---\\
Here an almost nude woman, wounded in the throat---

Wounded in the throat, unable to speak\\
Wounded in the lungs, unable to breath\\
Wounded in the breast, one breast\\
Lost---

Pierced through the breast\\
By javelin or scalpel\\
Hunter or surgeon\\
Or child

Uncomprehending, opening\\
A wound which does not heal:\\
Sutured crescent\\
Red    fierce   ragged:

An animal concealed in the forest\\
A body in the verdant\\
Body of nature\\
Whose wordless grief---

2.

Among the trees and shadows\\
Her shadowy children\\
Stand\\
Half-comprehending,

Unhealed---\\
Pierced\\
Their ragged hearts\\
Red   fierce

Torn\\
Here an almost nude woman\\
Wounded, mourned, \\
Lost---

3.

No.\\
No tapestry---\\
No mythological landscape---\\
No sad hounds, black with grief---

No blades\\
Of grass meticulously\\
Etched in soft tufts \\
Against her cheek

Nor minutely detailed\\
Dandelions: ``many-rayed\\
Yellow flowers'' ``Deeply\\
Notched basal leaves''

No bluets   no\\
Forget-me-nots\\
Pale\\
Blue-violet

Against the netted grass\\
And black\\
Earth.\\
No satyrs

With tufted ear\\
And half-comprehending grief,\\
No pale blue river diminishing\\
Into the formal landscape---

No\\
Paint, no rendering, no art\\
No surgery radical\\
Enough

To excise the silence\\
Of her wordlessness.

4.

Only the animal concealed in the forest\\
Only the animal veiled\\
Only the veil\\
Of our breath, our desire

Only desire\\
And loss.

5.

Still this is nothing. 
\end{verse}
