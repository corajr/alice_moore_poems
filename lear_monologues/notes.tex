\subsection{Lear: The butterfly shakes off not the crumpled cocoon (p. \pageref{ch:lear_aa})}
This is Lear's opening state of mind. There is (in his conception) a ``natural progression'', and it includes this peaceful \& orderly approach to death. Giving up power is purely ceremonial---he does not truly expect to be powerless, nor to give up the prerogatives he has always, always had as king. And he would not frame his expectation that Cordelia will care for him, center her life on him, even after marriage, as an expectation, but rather as a natural unfolding of the way things should be. Shakespeare's Lear refers to Cordelia's ``kind nursery'' but Lear does not conceive of this as some virtue of Cordelia's, an active principle in her, a strength: it is just a cog in the machinery of the natural world, and that natural world is ordered according to a hierarchy whose ultimate head may be God, but whose earthly head is the king. He sees nothing self-contradictory in phrases such as ``love's oaths,'' nor do the words ``The kingdom divided'' send a shudder down his spine, as they ought. As king, you divided \& apportion, you impose order, you maintain control.

He thinks.

\subsection{Cordelia: The rare holy day or feast (p. \pageref{ch:lear_ab})}
Cordelia's insights into her father's state of mind. And into her own relationship with him. As a child, her love \& innocence had sometimes relieved him of the burdens he carried as king. I imagine that her mother died in childbirth with her, or when she was very young; I also imagine that she is much younger than either of her two older sisters. Perhaps there were other children who died in infancy or childhood, but in any case, in his old age, Lear has just the three adult daughters. But even as Cordelia grew into a young woman, Lear wished to hold onto that child and the refuge of those moments of Edenic innocence. And perhaps, too, he wishes her to be a kind of wife-substitute, emotionally if not sexually. I can imagine Lear recast through the prism of some of the older, rawer, truly horrifying fairy tales in which the father harbors explicitly incestuous desires towards his daughter, and much being made of Lear's revulsion at mature female sexuality---but I don't see him in those terms. Goneril and Regan are not without insight into him, and R. remarks that ``he hath ever but slenderly known himself''; G. adds that the ``best and soundest of his time hath been but rash.'' With Cordelia, he imagines himself free to always see himself as the kindly father---but she dashes that daydream when she insists on answering his unanswerable question truthfully, and not in the language of a little girl who will gladly crown him anew every morning and lie to him that she loves her father ``all.'' ``Sure I shall never marry like my sisters,/To love my father all.''

\subsection{Lear: Survey my kingdom, map it, parish and pasture (p. \pageref{ch:lear_ac})}
Lear expects Cordelia to answer his call, fulfill his emotional needs, \& to do so with a delicacy and seeming freedom that the ugly tone of command throbbing beneath this opening ceremony completely contradicts. The reference here, in Lear's speech, to the Book of Daniel, chapter 5, is meant to be obvious: mene, mene, tekel, upharsin: numbered, numbered, weighed, divided: Lear at the opening of the play is all about commands \& measurement \& determining cause \& effect. He un--self-consciously echoes the Old Testament phrasing when he calls for the numbering of the fish, the measurement of all in kingdom (to the point of impossibility, but without acknowledging it---) so that it may be divided among his three daughters and their husbands. A division into three might have worked, just as a three-legged footstool may stand---but once he banishes Cordelia and divides her portion between the other two---each of whom would prefer to rule on her own---then he has destabilized the realm. And of course it could be argued that any division would have been fatal---he should have given the kingdom in its entirety to Cordelia? Perhaps---but you know that Goneril and Regan would not stand for that.

In the next two poems, Goneril and Regan make clear their understanding of their father. They may not love him, but they do know how to manipulate him, \& both see that he is foolishly susceptible to flattery.

\subsection{Goneril: Let my promises sound like love---let them (p. \pageref{ch:lear_ad})}

\subsection{Regan: When one cannot command, one flatters (p. \pageref{ch:lear_ae})}

The opening line here was originally Goneril's, but I realized that Regan would have an even keener understanding of its truth. She does not even have the power associated with being the eldest---nor is she the beloved ``baby'' of the family---you can be sure that any sisterly affection she had for Cordelia curdled ages ago.


\subsection{Cordelia: O swallow / Perplexing the eye in flight beyond scansion (p. \pageref{ch:lear_af})}
Cordelia is loving and sweet, but she's no fool. She sees her father clearly. She reads her sisters accurately, too.

Of course, in this play, the Fool's no fool, either.

\subsection{Kent: Eighty winters---some (p. \pageref{ch:lear_ag})}
Kent also sees the true nature of G\&R. He understands as Cordelia does that it is foolish to ask ``how much'' someone loves you, \& he tries to speak to Lear's error, trading on his position in the court \& in the king's affections; his stature \& privilege as a man \& a man of particular social class. But his essential claim, like Cordelia's, is that he loves Lear \& speaks in order to save him from himself.

``Orient myth'' is Kent's phrase---not from the play---but as I imagine him expressing himself. It reflects something old-fashioned in him. I see him as older, closer to the king in age; in contrast (for example) Edgar, in my mind, is much younger (he is Gloucester's son, after all), \& he wouldn't use the phrase.

\subsection{Regan: As for Cordelia--- (p. \pageref{ch:lear_ah})}
Regan mentions Lear's fear of ``nothing,'' \& speaks of him as ``Whittled, forked, stripped'': Lear, on seeing Mad Tom {[}Edgar{]}, says: ``unaccommodated man is no more but such a poor, bare, forked animal as thou art'': that ``forked animal'' has always stayed with me, as if Lear had a vision of ``man''---and therefore himself---as this elemental stick figure hardly more than a couple of cross strokes in water \& ash, scribbled on a scrap of paper.

\subsection{Edmund: Language becomes any animal I wish (p. \pageref{ch:lear_ai})}
The phrase ``i'th'heat'' is taken from Goneril's speech at the end of Act 1, scene 1---it's not Edmund's originally. But this monologue might be the concisest summation of the principles that unite this slippery trio---Edmund, Goneril, Regan: language doesn't have to communicate---it must serve the speaker, shape the speaker's reality; each believes in her or his own control of language; Edmund is confident (therefore) in his control over reality, G\&R want to believe in their control but each is uncertain of her hold on Edmund and that uncertainty unsettles their confidence; all three believe ``I want---it is enough.'' Now of course, Lear's always believed as much, but as king he's more or less always gotten what he wanted, so he hasn't had to connive \& betray \& murder. When he doesn't get his way, he throws a hissy fit. G\&R must take some delight in \emph{finally} being able to say, in effect, ``Screw you, old man, you don't get to come tromping through here with 50 ``friends'' whenever you effing want to.''

\subsection{The Fool: Untroubled by the alchemical giddiness of words (p. \pageref{ch:lear_aj})}
The Fool, at last. I love the Fool. He's not a fool, of course---Fools seldom are. It's been suggested that the same actor played the Fool as played Cordelia---they never speak on stage at the same time, and there's no indication that they appear together. It's almost as if the Fool is the repressed or silenced voice of Cordelia---what she would say if her role as daughter---as female---did not somehow make it impossible---either because it would not be proper (she thinks it, but does not say it) or because her conditioning as a good girl prevents her from even thinking it. My own ideas are closer to the former---she sees the truth clearly enough, but her desire to spare her father pain or embarrassment has kept her from speaking more plainly.

This monologue by the Fool is one of my favorites.

\subsection{Cornwall: You---boy---put aside your beer and beef and gossip (p. \pageref{ch:lear_ak})}
Cornwall has to insist on his power because he is uncertain of it. The awkward formality of his speech here---it's meant to sound vaguely like legalese---is intended to underline his reliance on the power structures that uphold his power. He does not have the force of personality of an Edmund, \& although he does not see the threat in Edmund, he does sense that his grasp on power isn't all it should be.The lord doth pontificate too much.

\subsection{Edmund: Reverend (p. \pageref{ch:lear_al})}
This is Edmund's silent address to his father, the Earl of Gloucester. The second and third lines of the second stanza are meant to be hard to deliver smoothly---Edmund is lecturing his father on his susceptibility to the sirens of flattery \& believing what you wish to believe---yet Edmund himself is susceptible to the sirens, as we will see, and the way those lines tend to trip you up even when you know they are coming (I trip up, and I wrote them!) is meant to reflect Edmund's own unconscious knowledge that he is lying to himself: he is not invulnerable, in any sense of the word, and his plans are no more impervious to disruption than the plans of any other mortal.

\subsection{Goneril and Regan: But then / It may be you are (p. \pageref{ch:lear_am})}
G\&R address this to Edmund, \& of course this is one of the most seductive songs of the sirens: you're different. You're not like the rest. You ``take away'' the breath of the siren: in effect, you special, special man, you silence and disempower her. All the rest? ``Dull as husbands.'' I was thinking of an exchange between Musetta and Marcello in Puccini's La Boheme when I wrote that line: they are exchanging insults \& this is one (essentially) that Musetta lobs at Marcello. I love the economy of this one, especially its ending: that beautiful vision, the incomparable song---all a lie, \& fatal at that.

\subsection{Edmund: With the vengefulness of a minor god (p. \pageref{ch:lear_an})}
Edmund doesn't want to contemplate the possibility that Regan killed her husband, not the reed of a boy who so quickly followed him into death. But this tells you of a weakness (as I see it) in Edmund's character: he thinks he sees the world with a cold, clinical eye, but the very idea of a Regan who is as capable of cold-blooded murder as he is---perhaps more so---\emph{that} he doesn't want to think about.

\subsection{Regan: My husband? (p. \pageref{ch:lear_ao})}
So Regan says she didn't do it. Probably she didn't. But it still isn't a good sign that Edmund won't even contemplate the possibility that she did. And yes, of course she intends to marry Edmund, now that he's an earl especially\ldots{} but that's probably just a formality in her mind.

\subsection{Edgar: Were I truly mad, how far away (p. \pageref{ch:lear_ap})}
References here to the language of the play, Edgar's in particular, but also to Tom O'Bedlam's Song, in the form collected in Untermeyer's anthology. Ironically, for one who plays the madman, \& therefore in the thinking of the time, the man who is most out of balance, most given to extremes, Edgar is at heart the most sturdy-hearted \& loving of the men. H. Granville-Barker draws upon one line in the play to characterize him: ``Bear free and patient thoughts,'' Edgar's counsel to his grief-maddened and blinded father. I think that one line stayed with me in all my thinking about Edgar, and every time I wrote a monologue for him.

\subsection{Fool: Protect us? (p. \pageref{ch:lear_aq})}
The Fool's speech here also reflects both Edgar's language (as Mad Tom) and Tom O'Bedlam's Song. But his rage is closer to the surface. The Fool is compassionate \& smart, like both Edgar \& Cordelia. But he is much freer spoken than either of them, \& his emotions are more vividly expressed.

\subsection{Cornwall: Her glance (p. \pageref{ch:lear_ar})}
Cornwall is dying, \& his senses are garbled \& fading. What he sees is therefore not entirely clear. Does Regan move too slowly to kill the rebellious servant? And is that slowness deliberate? Or does he imply that this ``boy'' was some ``toy'' of Regan's---politically? Sexually? If so, did she delay out of affection for her ``toy''? Or desire to hold onto a useful tool? Or did she not delay at all? Did it just seem that way to the dying Cornwall? He's always been a bit insecure\ldots{}now, does he die her toy?

\subsection{Albany: How could I not know her? Her own father (p. \pageref{ch:lear_as})}
Albany never did trust his wife. She acted the right way---but he was never convinced she meant it. And of course, if you set up a power structure where some people have access to power only through others, well, then, those people on the lower rungs learn what those further up the ladder want, \& give them the appearance of it, at least. Now \& then, though, you're bound to get some pretty ugly glimpses of what's hidden beneath the glittering surfaces.

\subsection{Gloucester: Too late / I saw their wolfish grins (p. \pageref{ch:lear_at})}

Like many of the good characters in Shakespeare's plays, Gloucester cannot see the true nature of the evil characters until it is too late. And like most of the good characters in Shakespeare's plays, it is not clear that Gloucester is wholly good: his attitude toward his bastard son is glib \& insensitive; his awareness of any wrong done to Edmund's mother is minimal if not missing entirely; his faith in the goodness of his ``legitimate'' son is shallow \& evaporates quickly; he thinks nothing of the fact that he was himself unfaithful to his marriage vows. But Gloucester, like Lear, is at least capable of seeing the error of some of his ways, \& admitting it before he dies.

\subsection{Kent: No, not wolves. Wolves (p. \pageref{ch:lear_au})}
Kent sees that the wolves at least live by some code, \& are faithful to that code. The lack of faithfulness to ties of natural affection \& hierarchy (as Kent would conceive of them) makes his enemies---\& the king's enemies---unnatural, \& all the more frightening \& repulsive for that.

\subsection{Albany: But when a man / Grins like a wolf (p. \pageref{ch:lear_av})}
Albany also sees unfolding events as representing a violation of the natural order.

\subsection{Lear: On the moor / In my own Hell (p. \pageref{ch:lear_aw})}
There are moments when Lear is totally out of it \& doesn't comprehend what is happening around him. Those moments are a welcome escape---\& relief---from his flashes of lucidity, summarized here.

\subsection{Gloucester: I have passed from light into darkness (p. \pageref{ch:lear_ax})}
As so often, in myth at least, the gods reveal the true nature of the world only once they have blinded the poor bastard they've subjected to their whims. That's what Gloucester feels here.

\subsection{Edmund: Wishes are for children--- (p. \pageref{ch:lear_ay})}
Edmund sneers at the Duke of Cornwall's rush to pluck out both of Gloucester's eyes, for in his view, allowing the duke to witness---\& contemplate---his own disfigurement would add a lovely frisson of ugly joy at the psychological anguish, now layered over what must be piercing physical torment. Even as he lingers on the thought of letting the engines of torture grind his enemies to dust, he celebrates the solidity of his own will.

\subsection{Lear: Too late / I knew their flattery (p. \pageref{ch:lear_az})}
An elaboration of the insights in his previous monologues.

\subsection{Edgar: Not a king, not a father, barely (p. \pageref{ch:lear_ba})}
In reviling, not just his elder daughters, but women, women's sexuality, Lear is reviling himself as well, for women are no more Other than any other man is Other. So ``Excrement,'' at first seems to apply to ``Woman,'' is in fact smeared all over him: that's what he's done to himself. But of course, as a powerful man---here, a king---what Lear does shapes---or misshapes---the world of others. G\&R are presented as monstrous---\& in some ways, I find them among Shakespeare's most horrifying characters, because of what they do to language, emptying it of meaning. And we see them when they are long past angling for power via the channels ordinarily available to women. They married powerful men, \& neither of them is wild over her partner, though it could be argued that at least Cornwall is a good match morally for Regan. By the end of the play, Goneril has betrothed herself to Edmund, who will have to devise some path to her bed via the murder of Albany; moreover, Goneril has poisoned Regan. But it's not much of a stretch to imagine that Lear has been leading everyone on a merry chase for some years now: he's 80 years old, more or less, \& used to getting his way. Even those who evidently love him---Cordelia, Kent, the Fool, possibly Gloucester---even they express concern over his rashness.

And yet a man---a person, any person---is always \& only just that stick figure, no more than ``a poor bare forked animal.'' It's one of the most essential truths of the play.

\subsection{Edmund: I will captain Death's (p. \pageref{ch:lear_bb})}
I think Edmund mostly addresses these monologues to himself. No one else is really worthy of them, in his mind. His desire is his primary rationale for any \& all action.

\subsection{Kent: The sky opened on pandemonium (p. \pageref{ch:lear_bc})}
Pandemonium is of course Milton's word. And ``shrapnel'' was a word unknown for more than two centuries after Shakespeare: it is derived from the name of Henry Shrapnel, 1761-1842, who invented it. But the words are just what I need, so I do not care if they are anachronisms.

\subsection{Gloucester: These holes that were my eyes (p. \pageref{ch:lear_bd})}
Gloucester must rethink his (re)evaluations of his sons.

\subsection{Edgar: In the autumn months, he comes (p. \pageref{ch:lear_be})}
Of course in the play, Edgar does not know about his father's blinding until they almost stumble upon each other. But here he appears to have heard what has happened. Or perhaps, reunited with him, he imagines his journey just previous. I don't think it matters.

\subsection{Oswald: Neither my mistress nor her sister (p. \pageref{ch:lear_bf})}
Oswald may not be smart, but he's cunning. I have stolen lines from Curran, from the beginning of Act 2, about ``likely wars . . . twixt'' duke \& duke, \& attributed them to Oswald. Oswald talks to himself---not because, like Edmund, he thinks he is the only worthy audience---but rather because he does not trust anybody. And Oswald sees Edmund more clearly than either sister does: Edmund ``worships a goddess'' {[}Nature{]} but loves no one. Of course, in truth, I think neither G nor R loves anyone, either, though, like Edmund, they wish to be ``belov'd.''

\subsection{Cordelia: I watch him burning, flowers in his hair (p. \pageref{ch:lear_bg})}

\subsection{Kent: Even so. He demands an answer, he demands (p. \pageref{ch:lear_bh})}
``Everyone he has known/ Transformed or gone'': that's Lear's Hell. Or at least one room in it.

\subsection{Edgar: But logic / Fails (p. \pageref{ch:lear_bi})}
Until Act 4, Lear tries to insist on imposing logic, most especially \emph{his} logic, on the world: everything must be conceived of in terms of cause \& effect.

\subsection{Cordelia: Wash his limbs (p. \pageref{ch:lear_bj})}

Cordelia knows who she is. Now Lear will waken and know her---\& begin to know why she answered as she did in Act 1.

``Fabula'' is a word from illuminated manuscript studies: it refers to all those strange creatures in the margins.

\subsection{Edmund: There is but one siren, and she sings to me (p. \pageref{ch:lear_bk})}
Of course Edmund has been fooled by the Siren. He thinks he is like her, even as he is seduced by her. A creature who sheds a scaly skin does not have ``skin . . . soft as down.'' But she is remorseless.

\subsection{Cordelia: If, in truth, we become God's spies (p. \pageref{ch:lear_bl})}
Cordelia is, like me, a creature of this world.

\subsection{Goneril: My sister! / That green-eyed (p. \pageref{ch:lear_bm})}
No love lost between G\&R. By the time Goneril delivers this monologue, Edmund is dying, Regan is dead, \& Goneril knows that one way or another, death awaits her. Her own rage consumes her---suicide offers not only the ``cooling dark'' of death, but imagined reunion with her beloved, Edmund---who never loved anyone. She imagines herself rather like Cleopatra going to her Antony---but of course neither of them ever approached the scope of achievement or the depth of feeling that those two colossi did.

\subsection{Regan: He was mine--- (p. \pageref{ch:lear_bn})}
Regan's clinging to the same sorry-ass Cleopatra fantasy---if anything, more explicitly than her sister.

\subsection{Albany: The battle is won. The day is lost. The King is dead. (p. \pageref{ch:lear_bo})}
An obvious reference to the account in the Venerable Bede of the sparrow in the banqueting hall. Albany is flawed, but struggles to do the right thing. The injustices of the world sting him, but he is late coming to the realization that he can do anything about them.

\subsection{Fool: Where the moat is overtaken by marsh still (p. \pageref{ch:lear_bp})}
After the events of the play, the Fool finds the Earl of Kent, who has seen his master through his troubles \& now, in the wake of the terrible deaths at the close of the play, he sees no more purpose in living. To continue to serve his master the king, he must follow him into the next world: ``My master calls me, I must not say no.''

\subsection{Edmund: Death consumes me (p. \pageref{ch:lear_bq})}
If there is anything not monstrous in Edmund---\& Shakespeare hints at as much at the close of the play when Edmund is moved to confess his plot on the lives of Lear \& Cordelia---then I hint at it here in Edmund's momentary relishing of the breeze ``once so sweet with rain.''

\subsection{The Fool: On the road past the blackened villages, (p. \pageref{ch:lear_br})}
The Fool doesn't really make Lear laugh much, does he? I don't think he's meant to. The Fool tells the truth when no one else is allowed to. Now the King he ``would not call King'' is dead \& so are all his daughters---but most piercingly, for the Fool, Cordelia is dead. He rides towards Death, but he tells us the truth on the way.

\subsection{Edgar: The starlings, with a cacophony almost (p. \pageref{ch:lear_bs})}
Edgar is the new Duke of Gloucester, but he feels little pleasure in the title. He has seen how shallow the roots of power grow, \& that power requires a brutality he is not capable of.

\subsection{Albany: Domine labia mea aperies (p. \pageref{ch:lear_bt})}
Years, decades later, the waste of those battles still haunts Albany.

\subsection{Edgar: Death approaches. (p. \pageref{ch:lear_bu})}
Many years after the close of the play, Edgar is near death himself now. But he has never stopped wondering just what Lear saw, knew, comprehended, in those last moments before his own death.

\subsection{Lear: No cause. No cause to burn on the wheel of fire (p. \pageref{ch:lear_bv})}
``No cause, no cause'' are Cordelia's words, in Act 4, scene 7, when Lear says to her that she has ``some cause'' to do him wrong: this is just what he must learn. Love is not a matter of cause \& effect, of measuring out what is owed or ``how much'' someone loves you.